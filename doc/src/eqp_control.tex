\begin{description}

\itt{error} is a scalar variable of type default \integer, that holds the
stream number for error messages. Printing of error messages in 
{\tt \packagename\_solve} and {\tt \packagename\_terminate} 
is suppressed if {\tt error} $\leq 0$.
The default is {\tt error = 6}.

\ittf{out} is a scalar variable of type default \integer, that holds the
stream number for informational messages. Printing of informational messages in 
{\tt \packagename\_solve} is suppressed if {\tt out} $< 0$.
The default is {\tt out = 6}.

\itt{print\_level} is a scalar variable of type default \integer, that is used
to control the amount of informational output which is required. No 
informational output will occur if {\tt print\_level} $\leq 0$. If 
{\tt print\_level} $= 1$, a single line of output will be produced for each
iteration of the process. If {\tt print\_level} $\geq 2$, this output will be
increased to provide significant detail of each iteration.
The default is {\tt print\_level = 0}.

%\itt{preconditioner} is a scalar variable of type default \integer, 
%that specifies which preconditioner to be used. When finding
%the required minimizer, the system matrix
%\disp{ \bmK_{H} = \mat{cc}{ \bmH & \bmA^T \\ \bmA  & \bmzero }}
%will be replaced by a so-called preconditioner of the form
%\eqn{prec1}{ \bmK_{G} = \mat{cc}{ \bmG & \bmA^T \\ \bmA  & \bmzero}
%\equiv \bmP \mat{ccc}{ \bmG_{11}^{} & \bmG_{21}^T & \bmA_1^T \\
%\bmG_{21}^{} & \bmG_{22}^{} & \bmA_2^T \\
%\bmA_{1}^{} & \bmA_{2}^{} & \bmzero} \bmP^T.}
%The leading-block matrix $\bmG$ will be a suitably-chosen 
%approximation to $\bmH$ for which $\bmK_{G}$ is easier to invert/factor\-ize
%than $\bmK_{H}$, while $\bmP$ is an appropriately-chosen permutation matrix.
%Possible values of {\tt preconditioner} are:

%\begin{description}
%\itt{0} the preconditioner is chosen automatically on the basis of which option 
%        looks likely to be the most efficient.
%\itt{1} $\bmG$ is chosen to be the identity matrix.
%\itt{2} $\bmG$ is chosen to be $\bmH$
%\itt{3} $\bmG$ is chosen to be the diagonal matrix whose diagonals
%        are the larger of those of $\bmH$ and a positive constant
%        (see {\tt min\_diagonal} below).
%\itt{4} $\bmG$ is chosen to be the band matrix  of given semi-bandwidth
%        whose entries coincide with those of $\bmH$ within the band.
%        (see {\tt semi\_bandwidth} below).
%\itt{11} $\bmG$ is chosen so that $\bmG_{11} = 0$, $\bmG_{21} = 0$
%        and $\bmG_{22} = \bmH_{22}$.
%\itt{12} $\bmG$ is chosen so that $\bmG_{11} = 0$, $\bmG_{21} = \bmH_{21}$
%        and $\bmG_{22} = \bmH_{22}$.
%\itt{-1} $\bmG$ is chosen so that $\bmG_{11} = 0$, $\bmG_{21} = 0$,
%        $\bmG_{22}$ is the identity matrix, and the preconditioner is computed
%        implicitly.
%\itt{-2} $\bmG$ is chosen so that $\bmG_{11} = 0$, $\bmG_{21} = 0$,
%        $\bmG_{22} = \bmH_{22}$ and the preconditioner is computed
%        implicitly.
%\end{description}
%Other values may be added in future.
%The default is {\tt preconditioner = 0}.

%\itt{min\_diagonal} is a scalar variable of type default \realdp, that 
%specifies the smallest value of the diagonal when diagonal preconditioning,
%{\tt precon = 3}, is used. The default is {\tt min\_diagonal = 0.00001}.

%\itt{semi\_bandwidth} is a scalar variable of type default \integer, that 
%specifies the semi-bandwidth of the band preconditioner when {\tt precon = 4}, 
%if appropriate.
%The default is {\tt semi\_bandwidth = 5}.

%\itt{factorization} is a scalar variable of type default \integer, that 
%specifies which factorization of the preconditioner should be used.  
%Possible values are:

%\begin{description}
%\itt{0} the factorization is chosen automatically on the basis of which option 
%        looks likely to be the most efficient.
%\itt{1} a Schur-complement factorization, involving factors of 
%       $\bmG$ and $\bmA \bmG^{-1} \bmA^T$, will be used.
%\itt{2} an augmented-system factorization, involving factors of $\bmK_G$,
%        will be used.
%\end{description}
%The default is {\tt factorization = 0}.

%\itt{max\_col} is a scalar variable of type default \integer, that specifies
%the maximum number of nonzeros in a column of $\bmA$ which is permitted
%by the Schur-complement factorization.
%The default is {\tt max\_col = 35}.

%\itt{indmin} is a scalar variable of type default \integer, that specifies
%an initial estimate as to the amount of integer workspace required by 
%the factorization package {\tt SILS}.
%The default is {\tt indmin = 1000}.

%\itt{valmin} is a scalar variable of type default \integer, that specifies
%an initial estimate as to the amount of real workspace required by 
%the factorization package {\tt SILS}.
%The default is {\tt valmin = 1000}.

%\itt{len\_ulsmin} is a scalar variable of type default \integer, that specifies
%an initial estimate as to the amount of workspace required by 
%the factorization package {\tt ULS}.
%The default is {\tt len\_ulsmin = 1000}.

%\itt{itref\_max} is a scalar variable of type default \integer, that specifies 
%the maximum number of iterative refinements allowed with each application 
%of the preconditioner.
%The default is {\tt itref\_max = 1}.

\itt{new\_h} is a scalar variable of type default \integer, that is used
to indicate how $\bmH$ has changed (if at all) since the previous call
to {\tt \packagename\_solve}. Possible values are:
\begin{description}
\itt{0} $\bmH$ is unchanged
\itt{1} the values in $\bmH$ have changed, but its nonzero structure 
is as before.
\itt{2} both the values and structure of $\bmH$ have changed.
\end{description}
The default is {\tt new\_h = 2}.

\itt{new\_a} is a scalar variable of type default \integer, that is used
to indicate how $\bmA$ has changed (if at all) since the previous call
to {\tt \packagename\_solve}. Possible values are:
\begin{description}
\itt{0} $\bmA$ is unchanged
\itt{1} the values in $\bmA$ have changed, but its nonzero structure 
is as before.
\itt{2} both the values and structure of $\bmA$ have changed.
\end{description}
The default is {\tt new\_a = 2}.

\itt{cg\_maxit} is a scalar variable of type default \integer, that is used
to limit the number of conjugate-gradient iterations performed in 
the optimality phase. If 
{\tt cg\_maxit} is negative, no limit will be impossed. The default
is {\tt cg\_maxit = 200}.

%\itt{pivot\_tol}  is a scalar variable of type default 
%\realdp, that holds the
%threshold pivot tolerance used by the matrix factorization.  See 
%the documentation for the packages {\tt SILS} and {\tt ULS} for details.
%The default is {\tt pivot\_tol = 0.01}.

%\itt{pivot\_tol\_for\_basis} is a scalar variable of type default 
%\realdp, that holds the
%threshold pivot  tolerance used by the package {\tt ULS} 
%when computing the non-singular basis matrix $\bmA_1$ for
%an implicit preconditioner. Since the calculation of a
%suitable basis is crucial, it is sensible to pick a larger
%value of {\tt pivot\_tol\_for\_basis} than of {\tt pivot\_tol}.
%The default is {\tt pivot\_tol\_for\_basis = 0.5}.

%\itt{zero\_pivot} is a scalar variable of type default \realdp, 
%that is used to detect singularity. Any pivot encountered during the
%factorization whose absolute value is less than or equal to  {\tt zero\_pivot}
%will be regarded as zero, and the matrix as singular.
%The default is {\tt zero\_pivot =} $u^{0.75}$
%where $u$ is {\tt EPSILON(1.0)} ({\tt EPSILON(1.0D0)} in 
%{\tt \fullpackagename\_double}).

\itt{radius} is a scalar variable of type default 
\realdp, that may be used to specify an upper bound on the norm
of the allowed solution (a ``trust-region'' constraint) during the
iterative solution of the optimality phase of the problem.
This is particularly useful if the
problem is unbounded from below. If {\tt radius} is set too small, there
is a possibility that this will preclude the package from 
finding the actual solution.
If {\tt initial\_radius} is not positive, it will be reset to the 
default value, {\tt initial\_radius = SQRT(0.1*HUGE(1.0))}
({\tt SQRT(0.1*HUGE(1.0D0))} in {\tt \fullpackagename\_double}).

%\itt{inner\_fraction\_opt} is a scalar variable of type default 
%\realdp, that specifies the
%fraction of the optimal value which is acceptable for the 
%iterative solution of the optimality phase of the problem using the package 
%{\tt \libraryname\_GLTR},
%and correspond to the value {\tt control\%inner\_fraction} in that package.
%A negative value is considered to be zero, and a value of larger than one
%is considered to be one. Reducing {\tt fraction\_opt} below one will result
%in a reduction of the computation performed at the expense of an inferior
%approximation to the optimal value.
%The default is {\tt inner\_fraction\_opt = 0.1}.

\itt{inner\_stop\_relative} and {\tt inner\_stop\_absolute} 
are scalar variables of type default \realdp, 
that hold the relative and absolute convergence tolerances for the
iterative solution of the optimality phase of the problem using the package 
{\tt \libraryname\_GLTR},
and correspond to the values {\tt control\%stop\_relative} and
{\tt control\%stop\_absolute} in that package.
The defaults are 
%{\tt inner\_stop\_relative = 0.0}
{\tt inner\_stop\_relative = 0.01}
and \sloppy {\tt inner\_stop\_absolute =} $\sqrt{u}$,
where $u$ is {\tt EPSILON(1.0)} ({\tt EPSILON(1.0D0)} in 
{\tt \fullpackagename\_double}).

\itt{max\_infeasibility\_relative} and {\tt max\_infeasibility\_absolute} 
are scalar variables of type default \realdp, 
that hold the relative and absolute tolerances for assessing 
infeasibility in the feasibility phase.  If the constraints are believed to be 
rank defficient and the norm of the residual $\bmA \bmx_T + \bmc$
at a "typical" feasiblke point is larger than
  $\max( ${\tt max\_infeasibility\_relative}$ \ast \| A\|,$ 
{\tt max\_infeasibility\_absolute}$)$,
the problem will be marked as infeasible,
The defaults are 
{\tt max\_infeasibility\_relative = 
\tt max\_infeasibility\_absolute =} $u^{0.75}$
where $u$ is {\tt EPSILON(1.0)} ({\tt EPSILON(1.0D0)} in 
{\tt \fullpackagename\_double}).

\itt{remove\_dependencies} is a scalar variable of type default \logical, 
that must be set \true\ if linear dependent constraints
$\bmA \bmx + \bmc = \bmzero$ should be removed and \false\ otherwise.
The default is {\tt remove\_dependencies = .TRUE.}.

%\itt{find\_basis\_by\_transpose} is a scalar variable of type default \logical, 
%that must be set \true\ if the invertible sub-block $\bmA_1$ of the 
%columns of $\bmA$ is computed by analysing the transpose of $\bmA$ 
%and \false\ if the analysis is based on $\bmA$ itself. Generally
%an analysis based on the transpose is faster.
%The default is {\tt find\_basis\_by\_transpose = .TRUE.}.

\itt{space\_critical} is a scalar variable of type default \logical, 
that must be set \true\ if space is critical when allocating arrays
and  \false\ otherwise. The package may run faster if 
{\tt space\_critical} is \false\ but at the possible expense of a larger
storage requirement. The default is {\tt space\_critical = .FALSE.}.

\itt{deallocate\_error\_fatal} is a scalar variable of type default \logical, 
that must be set \true\ if the user wishes to terminate execution if
a deallocation  fails, and \false\ if an attempt to continue
will be made. The default is {\tt deallocate\_error\_fatal = .FALSE.}.

\itt{FDC\_control} is a scalar variable of type 
{\tt FDC\_control\_type}
whose components are used to control any detection of linear dependencies
performed by the package 
{\tt \libraryname\_FDC}. 
See the specification sheet for the package 
{\tt \libraryname\_FDC} 
for details, and appropriate default values.

\itt{GLTR\_control} is a scalar variable argument of type 
{\tt GLTR\_control\_type} that is used to pass control
options to the conjugate-gradient solver used to solve linear systems that arise. 
See the documentation for the \galahad\ package {\tt GLTR} for further details. 
In particular, default values are as for {\tt GLTR}.

\itt{SBLS\_control} is a scalar variable argument of type 
{\tt SBLS\_control\_type} that is used to pass control
options to the symmetric block linear equation preconditioner 
used to help solve linear systems that arise. 
See the documentation for the \galahad\ package {\tt SBLS} for further details. 
In particular, default values are as for {\tt SBLS}.

\itt{prefix} is a scalar variable of type default \character\
and length 30, that may be used to provide a user-selected 
character string to preface every line of printed output. 
Specifically, each line of output will be prefaced by the string 
{\tt prefix(2:LEN(TRIM( prefix ))-1)},
thus ignoreing the first and last non-null components of the
supplied string. If the user does not want to preface lines by such
a string, they may use the default {\tt prefix = ""}.

\end{description}
